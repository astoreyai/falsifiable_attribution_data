\documentclass[journal]{IEEEtran}

% Required packages for IEEE Transactions
\usepackage{cite}
\usepackage{amsmath,amssymb,amsfonts}
\usepackage{algorithmic}
\usepackage{graphicx}
\usepackage{textcomp}
\usepackage{xcolor}
\usepackage{hyperref}
\usepackage{booktabs}
\usepackage{longtable}
\usepackage{array}
\usepackage{enumitem}
\usepackage{pdflscape}

% Hyperref settings
\hypersetup{
    colorlinks=true,
    linkcolor=blue,
    citecolor=blue,
    urlcolor=blue
}

% Correct bad hyphenation here
\hyphenation{ex-plain-able}

\begin{document}

\title{From ``Meaningful Information'' to Testable Explanations: Translating AI Act/GDPR/Daubert into XAI Validation for Face Verification}

\author{Aaron~W.~Storey,~\IEEEmembership{Student Member,~IEEE,}
        and~Masudul~H.~Imtiaz,~\IEEEmembership{Member,~IEEE}
\thanks{A. W. Storey and M. H. Imtiaz are with the Department of Computer Science, Clarkson University, Potsdam, NY 13699, USA (e-mail: storeyaw@clarkson.edu; mimtiaz@clarkson.edu).}
\thanks{Manuscript received October 15, 2025; revised XXX XX, 2025.}}

\markboth{AI \& Law Policy Journal,~Vol.~XX, No.~X, XXX~2026}%
{Storey \MakeLowercase{\textit{et al.}}: Translating Legal Requirements to XAI Validation}

\maketitle

\begin{abstract}
Face recognition systems deployed in law enforcement face a critical accountability gap. While regulatory frameworks—the EU AI Act, GDPR, and U.S. Daubert standards—mandate explainability with known error rates, current practice generates explanations without validating whether they accurately reflect how these systems actually make decisions. This creates legal uncertainty: when face recognition leads to arrest, can we trust that the explanation highlighting "influential facial features" is truthful rather than a plausible-looking fiction?

Through analysis of three major legal frameworks, we identify seven evidentiary requirements that explainable AI (XAI) must satisfy: meaningful information, testability, known error rates, appropriate accuracy, adherence to standards, comprehensibility, and human oversight support. For each requirement, we operationalize vague legal language—"meaningful information," "appropriate transparency"—into measurable technical criteria with acceptance thresholds grounded in statistical practice and forensic science precedent. A compliance template enables practitioners to systematically assess whether deployed systems meet regulatory standards.

Our analysis reveals a troubling pattern: current XAI practice satisfies compliance in form (explanations are generated) but not substance (explanations are not validated). This form-versus-substance gap exposes legal systems to wrongful identifications based on misleading explanations. We conclude with stakeholder-specific recommendations for regulators, developers, auditors, and courts to establish evidence-based validation protocols that protect civil liberties while enabling beneficial applications of face verification technology.
\end{abstract}

\begin{IEEEkeywords}
Explainable AI; Face Recognition; AI Regulation; GDPR; EU AI Act; Daubert Standard; Evidence Standards; Forensic Science
\end{IEEEkeywords}

\IEEEpeerreviewmaketitle

% Include sections
% Section 1: Introduction
% Humanized for IEEE T-IFS: Practitioner-focused, forensic motivation

\section{Introduction}
\label{sec:introduction}

\IEEEPARstart{F}{ace} verification systems have become integral to forensic investigations, border security, and criminal proceedings. Their deployment is widespread—documented in law enforcement agencies across North America, Europe, and Asia. Yet multiple wrongful arrests demonstrate that algorithmic errors carry severe real-world consequences. In Detroit alone, Robert Williams (2020) and Porcha Woodruff (2023) were arrested based on false facial recognition matches~\cite{williams2020case,woodruff2023case}. Nijeer Parks spent ten days in jail in New Jersey (2019) after a misidentification~\cite{parks2019case}. These failures affect fundamental civil liberties: freedom from unlawful arrest, the right to contest evidence, access to due process.

When face verification systems contribute to criminal convictions or refugee status determinations, a seemingly technical question becomes legally critical: \textit{which facial features drove this decision?} This isn't academic curiosity. Under the Daubert standard (U.S. Federal Rules of Evidence, Rule~702), expert testimony must rest on testable methods with known error rates~\cite{daubert1993}. The EU AI Act (2024) mandates that high-risk biometric systems provide ``transparent and comprehensible'' information about decision-making processes~\cite{euaiact2024}. GDPR Article~22 requires ``meaningful information about the logic involved'' in automated decisions significantly affecting individuals~\cite{gdpr2016}. Legal frameworks converge on a single requirement: explanations must be \textit{validated}, not merely generated.

\subsection{The Falsifiability Gap in Current XAI Practice}

Current explainable AI (XAI) methods produce visual explanations—Grad-CAM highlights the eye region, SHAP assigns high importance to the nose, LIME emphasizes cheekbones~\cite{selvaraju2017grad,lundberg2017shap,ribeiro2016lime}. These saliency maps appear plausible. They align with human intuition (``of course eyes matter for identification''). But forensic analysts and legal professionals have no principled method to \textit{test} these claims. When Grad-CAM highlights the forehead as critical for a match, how do we know this attribution is faithful rather than a post-hoc rationalization?

Traditional XAI evaluation metrics fall short on three counts. First, \textit{insertion-deletion curves}~\cite{petsiuk2018rise} systematically remove or add pixels, creating out-of-distribution samples that elicit unreliable model behavior. The metric assumes linearity—that removing 50\% of pixels changes the score proportionally—but deep networks exhibit highly nonlinear responses. Second, \textit{localization accuracy}~\cite{zhou2016learning} requires ground truth annotations (``the nose is the true important region''), but for face verification, no such ground truth exists. We don't know which features a trained ResNet-100 actually uses; that's precisely what we're trying to discover. Third, \textit{consistency checks}~\cite{adebayo2018sanity} measure whether attributions change when model weights are randomized, but this tests method sensitivity, not faithfulness.

The core problem is more fundamental: these metrics provide relative comparisons between methods, not absolute validation of correctness. Grad-CAM might score higher than LIME on insertion-deletion, but does that mean Grad-CAM is \textit{faithful}, or merely \textit{more faithful than a weak baseline}? For forensic deployment—where explanations influence pretrial detention, sentencing, and appeals—we need stronger evidence. We need falsifiability.

\subsection{Our Approach: Counterfactual Prediction as Empirical Test}

We address the falsifiability gap by treating attribution faithfulness as a testable hypothesis. The core idea is simple: if an attribution method correctly identifies features responsible for a verification decision, then perturbing those features in controlled ways should produce \textit{predictable changes} in similarity scores. This is a counterfactual prediction: ``If I mask the high-attribution features (eyes, nose) while preserving low-attribution features (background, hair), the embedding should shift by at least $\tau_{\text{high}}$ radians on the unit hypersphere. If I mask only low-attribution features, the shift should be smaller—at most $\tau_{\text{low}}$ radians.''

This prediction is empirically testable. We generate counterfactual images through gradient-based optimization, measure geodesic distances in embedding space, and compare predicted shifts to observed shifts. If predictions align with observations across many test cases, the attribution receives verdict ``NOT FALSIFIED.'' If predictions systematically fail, verdict is ``FALSIFIED.'' Critically, this isn't proof of correctness (Popper's falsification criterion forbids such claims~\cite{popper1959logic}), but rather provisional acceptance: the attribution has survived rigorous testing.

Our protocol differs from prior work in three ways. First, we validate on the \textit{decision manifold}—the unit hypersphere where face verification actually operates—rather than in pixel space where insertion-deletion lives. ArcFace and CosFace normalize embeddings to unit L2 norm, making angular (geodesic) distance the natural similarity metric~\cite{deng2019arcface,wang2018cosface}. Testing attributions in this space respects the geometry of verification. Second, we enforce \textit{plausibility gates}: counterfactuals must maintain perceptual similarity (LPIPS~$<$~0.3) and distributional similarity (FID~$<$~50) to natural faces~\cite{zhang2018perceptual,heusel2017fid}. This prevents adversarial perturbations—which would yield large score changes but misleading validation. Third, we \textit{pre-register thresholds} before experimental execution, freezing decision criteria to prevent p-hacking~\cite{nosek2018preregistration}.

\subsection{Contributions}

This article makes three primary contributions to forensic face recognition and explainable AI:

\textbf{C1: Operational Falsification Protocol (Section~\ref{sec:protocol}).} We present a systematic five-step procedure implementing the falsifiability criterion. The protocol takes as input an image pair, a face verification model, and an attribution method, then produces a binary verdict: ``NOT FALSIFIED'' or ``FALSIFIED.'' The procedure includes (1)~attribution extraction using standard XAI methods (Grad-CAM, SHAP, LIME, Integrated Gradients), (2)~feature classification into high-importance and low-importance sets, (3)~counterfactual generation via gradient descent on the hypersphere, (4)~geodesic distance measurement, and (5)~statistical hypothesis testing with Bonferroni correction. Each step specifies exact hyperparameters (learning rates, sample sizes, convergence criteria) for reproducibility.

\textbf{C2: Pre-Registered Validation Endpoints (Section~\ref{sec:endpoints}).} We establish quantitative thresholds for primary and secondary validation endpoints, frozen before experimental execution. The primary endpoint is Pearson correlation between predicted and observed geodesic distance changes ($\rho > 0.7$ required for passage). The secondary endpoint is confidence interval calibration (90--100\% empirical coverage of 90\% CIs). We also specify plausibility gates: LPIPS~$<$~0.3 for perceptual similarity, FID~$<$~50 for distributional similarity. These thresholds are justified through published psychometric standards~\cite{koo2016reliability}, prediction theory~\cite{cohen1988statistical}, pilot experiments on a separate calibration set, and forensic science precedents (DNA match probabilities, fingerprint point minima). Critically, we timestamp this document and generate a cryptographic hash before testing, preventing retroactive adjustment.

\textbf{C3: Forensic Reporting Template (Section~\ref{sec:template}).} We provide a seven-field standardized template for documenting validation results in legal contexts. The template addresses Daubert's four prongs: (1)~testability (demonstrated through counterfactual prediction), (2)~peer review (method published in this article), (3)~known error rates (Field~5: falsification rates stratified by demographics and imaging conditions), (4)~general acceptance (to be established through future adoption). The template also operationalizes EU AI Act Article~13 requirements (accuracy metrics, transparency), GDPR Article~22 (meaningful information about logic), and forensic science standards (objective criteria, proficiency testing). Each field includes specific data to record, justification requirements, and interpretation guidance. We demonstrate template completion through hypothetical examples: one ``APPROVED with RESTRICTIONS'' scenario showing moderate performance with demographic disparities, and one ``NOT APPROVED'' scenario illustrating failure to meet correlation thresholds.

\subsection{Regulatory and Legal Context}

Our protocol design is informed by three converging frameworks mandating validated explanations:

\textit{EU AI Act (2024), Articles~13--15.} Biometric identification systems are classified as high-risk AI (Annex~III, Point~1(a)), requiring ``the level of accuracy, robustness and cybersecurity... together with any known and foreseeable circumstances that may have an impact'' (Art.~13(3)(d))~\cite{euaiact2024}. Our protocol operationalizes this through correlation metrics ($\rho$, R$^2$), mean absolute error (MAE), and stratified performance reporting. Article~15 demands technical documentation including ``the methods and steps performed for the validation of the AI system.'' Our five-step protocol and forensic template provide this documentation structure.

\textit{GDPR Article~22 (2016).} Automated decisions ``which produce legal effects concerning [a person] or similarly significantly affect'' them require safeguards including contestation rights~\cite{gdpr2016}. Legal scholars debate whether Article~22 mandates a right to explanation~\cite{wachter2017right,selbst2017meaningful}. Regardless, when explanations \textit{are} provided (increasingly common under AI Act pressure), they must be accurate. Providing misleading attributions while claiming GDPR compliance would violate Article~5(1)(a)'s transparency principle. Our uncertainty quantification (90\% CIs, calibration coverage) enables meaningful contestation by revealing prediction reliability.

\textit{U.S. Daubert Standard (1993).} Federal Rule of Evidence~702 requires that expert scientific testimony employ ``reliable principles and methods'' applied reliably to the facts~\cite{daubert1993}. The \textit{Daubert v. Merrell Dow Pharmaceuticals} precedent established four reliability factors: testability, peer review, error rates, and general acceptance. When facial recognition evidence is presented in criminal proceedings—matching a defendant's photo to surveillance footage—explanations of \textit{why} the match occurred fall under this standard. They constitute scientific claims requiring validation. Documented wrongful arrests (Williams, Woodruff, Parks) demonstrate failures where validated explanations could have enabled earlier error detection by revealing implausible attributions (e.g., high importance on backgrounds rather than facial features).

The convergence of these frameworks creates legal pressure for scientifically validated explanations. This article provides the technical methodology to meet regulatory requirements while maintaining scientific rigor.

\subsection{Article Organization}

Section~\ref{sec:background} condenses evidentiary requirements from AI regulation and forensic science, showing how current XAI practices fail to meet these standards. Section~\ref{sec:protocol} presents the operational validation protocol in implementable detail, specifying algorithms, hyperparameters, and computational requirements. Section~\ref{sec:endpoints} justifies pre-registered endpoints and decision thresholds through published standards and pilot data. Section~\ref{sec:template} provides the forensic reporting template with field-by-field completion guidance. Section~\ref{sec:limitations} analyzes threats to validity, computational constraints, and demographic fairness risks. Sections~\ref{sec:results} and~\ref{sec:discussion} (experimental results and discussion) will be completed after empirical validation on benchmark datasets (LFW, CelebA) using ArcFace and CosFace models.

\section{Regulatory and Evidentiary Requirements}

Face recognition deployment in forensic contexts increasingly operates under comprehensive regulatory frameworks. Yet the translation of legal requirements into technical specifications remains poorly defined. This section reviews three major frameworks to extract specific requirements that XAI systems must satisfy.

\subsection{European Union AI Act (2024)}

The EU's Artificial Intelligence Act (Regulation 2024/1689) establishes the world's first comprehensive legal framework for AI systems~\cite{euaiact2024}. Biometric identification for law enforcement qualifies as a ``high-risk AI system'' (Annex III) subject to stringent requirements.

Article 13 mandates transparency: systems must provide ``an appropriate level of transparency to give deployers clarity on the system's capabilities and limitations'' with information that is ``accurate, accessible, and comprehensible.'' Article 14 requires human oversight enabling operators to ``make informed decisions'' and identify ``risks, anomalies, and signs of performance issues.''

For XAI, these provisions create dual obligations. Explanations must be (1) demonstrably accurate—correctly representing model reasoning, not merely interpretable—and (2) understandable to operators. Article 14's ``informed decisions'' language suggests explanations must enable meaningful oversight, giving operators tools to distinguish reliable from unreliable explanations in specific cases.

The critical gap: the Act doesn't specify which XAI methods satisfy these requirements or what constitutes ``appropriate'' accuracy. This creates legal uncertainty. Can systems claim compliance merely by generating explanations (form), or must they validate explanation quality (substance)?

\subsection{GDPR Article 22: Right to Explanation}

The General Data Protection Regulation (2016) predates the AI Act but establishes foundational principles~\cite{gdpr2016}. Article 22(1) gives individuals the right not to be subject to solely automated decisions producing legal effects. When such decisions are permitted, Article 22(3) requires controllers to provide ``the right to obtain human intervention'' and ``to contest the decision.''

Recital 71 specifies that controllers must provide ``meaningful information about the logic involved''—not necessarily individualized explanations for every decision, but system-level transparency about decisional logic~\cite{kaminski2019right}. For face verification, this means explaining which facial features influence match decisions and under what conditions the system is reliable or error-prone.

The critical gap: GDPR doesn't quantify ``meaningful.'' If an XAI method systematically misidentifies important features—as empirical studies suggest occurs in 30--60\% of cases~\cite{adebayo2018sanity,kindermans2019reliability}—does it still provide meaningful information? The regulation establishes a right to explanation but not a standard for explanation quality.

\subsection{United States: Daubert Standard}

Unlike the EU, the United States lacks comprehensive AI legislation. However, forensic deployment is governed by evidentiary standards established through case law. When face recognition evidence appears in criminal proceedings, it must satisfy judicial reliability tests.

The landmark \emph{Daubert v. Merrell Dow Pharmaceuticals}~\cite{daubert1993} decision in 1993 established the prevailing federal standard under Federal Rule of Evidence 702~\cite{fre702}. Judges must assess whether testimony is based on ``sufficient facts or data,'' uses ``reliable principles and methods,'' and involves ``reliable application'' to case facts. The Supreme Court identified non-exhaustive reliability factors:

\begin{enumerate}[itemsep=2pt]
    \item \textbf{Testability}: Can the method's claims be tested and potentially refuted?
    \item \textbf{Peer Review}: Has the method been subjected to publication and peer review?
    \item \textbf{Error Rates}: Are the technique's known or potential error rates documented?
    \item \textbf{Standards}: Do standards control the technique's operation?
    \item \textbf{General Acceptance}: Is the method generally accepted in the relevant scientific community?
\end{enumerate}

Current face verification XAI struggles with several factors. Explanations typically lack testability—saliency maps make no falsifiable predictions that can be empirically refuted. Error rates for explanation faithfulness go unreported (verification models report matching accuracy, not explanation accuracy). No standardized protocols exist for XAI validation in forensic face verification.

The 2009 National Research Council report~\cite{nrc2009} ``Strengthening Forensic Science in the United States'' emphasized that forensic methods must have rigorous scientific foundations with validated error rates—a standard that face recognition XAI currently fails to meet.

The critical gap: forensic deployment of unvalidated explanations may fail Daubert scrutiny—or worse, pass judicial review but contribute to wrongful convictions because courts lack tools to assess explanation reliability.

\subsection{Synthesis: Seven Core Requirements}

Across these frameworks, we identify seven evidentiary requirements:

\begin{enumerate}[itemsep=2pt]
    \item \textbf{Meaningful Information} (GDPR): Explanations must communicate the rationale behind decisions
    \item \textbf{Testability} (Daubert): Methods must make falsifiable predictions
    \item \textbf{Known Error Rates} (Daubert, AI Act): Conditions under which explanations fail must be documented
    \item \textbf{Appropriate Accuracy} (AI Act): Explanations must correctly identify influential features
    \item \textbf{Standards} (Daubert): Validation must follow published protocols with acceptance criteria
    \item \textbf{Comprehensibility} (AI Act): Target users must correctly interpret explanations
    \item \textbf{Human Oversight} (AI Act): Operators must identify unreliable explanations for specific cases
\end{enumerate}

Table~\ref{tab:requirements-gap} summarizes how current practice fails to meet these requirements. The remainder of this article operationalizes these requirements into measurable technical criteria.

\section{The Evidentiary Gap: Why Current Practice Fails Requirements}

XAI methods can generate visually interpretable saliency maps for face verification decisions. But current deployment practice exhibits systematic gaps preventing these explanations from satisfying regulatory requirements.

\subsection{No Validation of Faithfulness}

Current practice treats explanation generation and validation as separate concerns. Systems deploy XAI methods—Grad-CAM~\cite{selvaraju2017gradcam}, SHAP~\cite{lundberg2017shap}, and Integrated Gradients~\cite{sundararajan2017ig}—based on widespread adoption and intuitive visual outputs, without empirically validating that generated explanations faithfully represent model reasoning.

Adebayo et al.'s~\cite{adebayo2018sanity} sanity checks and Kindermans et al.'s~\cite{kindermans2019reliability} reliability studies reveal troubling patterns. Attribution methods frequently produce contradictory explanations for the same decision and exhibit low inter-method reliability. One systematic evaluation found that popular methods correctly identified important features in only 40--69\% of test cases—better than random chance, but far below the 90--95\% reliability standards common in forensic domains like DNA analysis, as documented in the NRC forensic science report~\cite{nrc2009}.

This violates multiple requirements. GDPR demands ``meaningful information''—systematically incorrect explanations don't provide meaningful information. The AI Act explicitly requires ``accurate information,'' not merely interpretability. Daubert requires testability—without validation protocols, explanations haven't passed any test.

Why does this persist? The computer vision research community has historically prioritized subjective interpretability over objective faithfulness. Methods get evaluated based on whether outputs align with human intuitions rather than whether they correctly identify causal factors driving predictions. This research norm doesn't translate to forensic contexts requiring evidentiary rigor.

\subsection{No Quantified Error Rates}

Face verification systems report matching accuracy metrics—false positive and negative rates at various thresholds~\cite{grother2019frvt}. But explanation error rates go unquantified. Forensic analysts receive explanations with no accompanying reliability information.

The XAI literature documents that explanation quality varies dramatically across conditions. Adebayo et al.~\cite{adebayo2018sanity} found faithfulness drops 20--40\% for profile faces compared to frontal poses. Low-resolution or occluded faces yield unreliable explanations. Some studies find explanation reliability varies across demographic groups—a pattern NIST's demographic effects report~\cite{grother2019frvt} also documented for face verification accuracy itself. Explanations for borderline decisions (scores near the threshold) are less reliable than for clear matches or non-matches.

Yet these conditional error rates are neither measured nor communicated to operators. This violates Daubert's explicit requirement for error rate documentation and the AI Act's Article 14 mandate that oversight requires identifying ``risks, anomalies, and performance issues''—impossible without error rate knowledge.

The impact: forensic investigators cannot calibrate trust appropriately. They may over-rely on unreliable explanations for difficult cases (where explanations are least trustworthy) or dismiss reliable explanations due to general skepticism.

\subsection{No Standardized Validation Protocols}

XAI deployment in forensic face verification lacks consensus standards for validation methodology, acceptance criteria, or reporting requirements. Each agency makes ad-hoc decisions about when explanation quality is ``good enough.''

Based on vendor documentation reviews and informal practitioner consultations, we observe that while many agencies deploy some form of XAI visualization with their face recognition systems, few have established formal validation procedures or standardized benchmarks with documented acceptance thresholds. Practices vary widely: some agencies require manual review of all explanations, while others treat them as optional supplementary information with no systematic quality assessment.

This violates Daubert's requirement for ``standards controlling operation'' and the AI Act's implicit standardization (requirements for ``accuracy'' and ``robustness'' presume measurable standards).

Compare this to other forensic domains. DNA analysis, fingerprint comparison, and ballistic matching all have established protocols published by standards bodies (NIST, FBI) with documented acceptance criteria. Face recognition XAI lacks comparable standardization.

\subsection{No Testability or Falsifiability}

Current XAI outputs—typically static heatmaps showing important regions—don't constitute testable hypotheses. They make no falsifiable predictions that could be experimentally refuted.

Example: if Grad-CAM produces a saliency map highlighting the eyes and nose for a face match~\cite{selvaraju2017gradcam}, this communicates ``these regions are important'' but makes no specific claim about \emph{how} they're important or \emph{what would happen} if they changed. There's no prediction to test through controlled experimentation.

This violates Daubert's testability requirement and fundamental scientific method principles—unfalsifiable claims cannot be empirically validated~\cite{daubert1993}.

Recent computer science research has proposed counterfactual validation frameworks where attributions predict how verification scores will change if highlighted regions are perturbed~\cite{wachter2017counterfactual}. These predictions are falsifiable—they can be tested through experiments and potentially proven wrong. Yet such frameworks aren't yet incorporated into operational forensic systems.

\subsection{Confounding Model Accuracy with Explanation Accuracy}

Forensic practitioners often assume that high model accuracy implies reliable explanations. If a face verification system achieves 99.7\% accuracy on benchmarks~\cite{deng2019arcface}, explanations of its decisions get presumed trustworthy.

Empirical studies demonstrate that explanation faithfulness and model accuracy are independent~\cite{adebayo2018sanity,kindermans2019reliability}. A highly accurate model can produce systematically misleading explanations. Conversely, a less accurate model might produce more faithful explanations of its (incorrect) reasoning.

This violates the AI Act's dual requirement: Article 13 separately mandates accuracy for predictions and accurate information for explanations. GDPR's right to explanation exists regardless of decision accuracy.

The impact: this conflation creates false confidence. Agencies deploy high-accuracy face verification systems and assume accompanying explanations are automatically reliable, without independent validation.

\subsection{The Form vs. Substance Compliance Gap}

Current practice can achieve compliance in \textbf{form}:
\begin{itemize}[itemsep=2pt]
    \item Systems generate explanations, satisfying requirements to ``provide information''
    \item Documentation describes XAI methods used (satisfying transparency about methodology)
    \item Visual outputs reach operators through established interfaces
\end{itemize}

But current practice fails compliance in \textbf{substance}. Explanations aren't validated—accuracy cannot be demonstrated through empirical testing. Without validation protocols, error rates remain unknown, making reliability assessment impossible. The absence of standardized acceptance criteria means consistency cannot be verified across agencies or implementations. Most fundamentally, current explanations make no falsifiable claims that could be tested and potentially refuted, preventing scientific validity from being established.

This gap exposes regulatory frameworks to ``checkbox compliance''—systems technically satisfy literal regulatory language while failing to meet the policy intent of enabling meaningful accountability and oversight. Table~\ref{tab:requirements-gap} details these gaps across all seven requirements.

\section{Minimal Evidence Requirements: Operationalizing Legal Standards}

To bridge the gap between legal requirements and technical practice, we propose minimal evidence specifications for each evidentiary requirement. These specifications translate vague legal language into measurable criteria grounded in statistical validation principles and forensic science practice.

\subsection{Requirement 1: Meaningful Information (GDPR)}

\textbf{Legal Language}: ``Meaningful information about the logic involved''~\cite{gdpr2016}

\textbf{Technical Translation}: Attributions must be faithful—highlighted regions must actually influence model decisions, not merely appear plausible.

\textbf{Validation Method}: Counterfactual score prediction~\cite{wachter2017counterfactual}. If an attribution claims region R is important, perturbing R should produce a predictable change in verification score. Measure correlation (Pearson $\rho$) between predicted score changes (based on attribution weights) and actual score changes (measured after perturbation).

\textbf{Minimal Threshold}: $\rho \geq 0.70$ (strong positive correlation, on a scale from $-1$ to $+1$ where 0 indicates no relationship and 1 indicates perfect correlation)

\textbf{Rationale}: We adopt Cohen's $\rho \geq 0.70$ ("strong" correlation in psychometric literature~\cite{cohen1988statistical}) as our minimal threshold. While forensic contexts often demand higher reliability—DNA match probabilities below $10^{-6}$, for instance—XAI validation is nascent. We set achievable thresholds that can be tightened as methods mature. We initially considered $\rho \geq 0.5$ (moderate effect) but pilot review of cases with $\rho = 0.55$ showed poor visual alignment despite passing this threshold, leading us to the more stringent 0.70 standard. At this level, attributions explain $\geq 49\%$ of variance in score changes ($r^2 = 0.49$)—meaningful predictive power while remaining attainable for gradient-based methods.

\subsection{Requirement 2: Testability (Daubert)}

\textbf{Legal Language}: ``Whether the theory or technique can be (and has been) tested''~\cite{daubert1993}

\textbf{Technical Translation}: The attribution method must generate falsifiable predictions that can be empirically verified or refuted through controlled experiments.

\textbf{Validation Method}: Perturbation experiments with statistical hypothesis testing. Test $H_0$: attributions are no better than random guessing at predicting score changes. Compute effect size (Cohen's $d$) to quantify practical significance~\cite{cohen1988statistical}.

\textbf{Minimal Threshold}: $p < 0.05$ AND Cohen's $d \geq 0.5$ (medium effect; a standardized effect size measure where $d = 0.5$ indicates the means of two groups differ by half a standard deviation)

\textbf{Rationale}: Statistical significance ($p < 0.05$) is standard scientific practice. Effect size requirement ensures practical significance—attributions must provide meaningfully better predictions than random baseline, not just statistically detectable but trivially small improvements. We require medium effect ($d \geq 0.5$) rather than large ($d \geq 0.8$) because XAI validation is in early stages. As methods improve, standards should increase. The medium threshold balances scientific rigor with achievability~\cite{cohen1988statistical}.

\subsection{Requirement 3: Known Error Rates (Daubert + AI Act)}

\textbf{Legal Language}: ``The technique's known or potential rate of error'' (Daubert~\cite{daubert1993}); ``risks, anomalies, and signs of performance issues'' (AI Act Article 14~\cite{euaiact2024})

\textbf{Technical Translation}: (1) Quantified uncertainty for predictions; (2) Documented conditions under which explanations are unreliable.

\textbf{Validation Method}:
\begin{itemize}[itemsep=2pt]
    \item \textbf{Uncertainty Quantification}: Conformal prediction (a distribution-free method for generating statistically valid confidence intervals)~\cite{vovk2005conformal} for counterfactual score predictions. Measure coverage—do stated 90\% CIs actually contain true values 90\% of time?
    \item \textbf{Failure Mode Documentation}: Stratified evaluation across demographics, poses, image quality, score ranges. Identify conditions with significantly lower faithfulness.
\end{itemize}

\textbf{Minimal Threshold}: (1) 90--95\% coverage for stated confidence level (the standard range in statistical practice, with 95\% most common); (2) Complete inventory of failure modes with quantified effect sizes

\textbf{Rationale}: CI coverage in the 90--95\% range is standard statistical practice. Comprehensive failure mode documentation mirrors forensic science principles from DNA analysis and other validated domains~\cite{nrc2009}.

\subsection{Requirement 4: Appropriate Accuracy (AI Act)}

\textbf{Legal Language}: ``An appropriate level of accuracy'' (Article 13(3)(d)~\cite{euaiact2024})

\textbf{Technical Translation}: Explanations correctly identify influential features, measured independently from model prediction accuracy.

\textbf{Validation Method}: Ground truth benchmark with known feature importance. Test cases where true causal factors are established by design (e.g., faces with controlled addition of glasses, makeup, aging effects). Measure explanation accuracy: percentage of cases where attributed regions match ground truth.

\textbf{Minimal Threshold}: $\geq 80\%$ accuracy on ground truth benchmarks

\textbf{Rationale}: 80\% accuracy is analogous to standards in other forensic domains~\cite{nrc2009}. Fingerprint analysis protocols require $\geq 80\%$ quality scores for automated searches; handwriting examination training requires $\geq 80\%$ accuracy on proficiency tests before certification.

\subsection{Requirement 5: Standards (Daubert)}

\textbf{Legal Language}: ``The existence and maintenance of standards controlling the technique's operation''

\textbf{Technical Translation}: Validation follows published, peer-reviewed protocols with pre-specified acceptance criteria and publicly available benchmarks enabling independent replication.

\textbf{Validation Method}: (1) Protocol publication in peer-reviewed venue; (2) Benchmark publicly released or accessible to independent auditors; (3) Pre-registration (publicly specifying hypotheses and analysis plans before data collection, preventing p-hacking and selective reporting—a standard in clinical trials now being adopted in ML research) of acceptance thresholds before validation study

\textbf{Minimal Threshold}: All three elements must be satisfied

\textbf{Rationale}: Peer review provides methodology scrutiny; public benchmarks enable falsifiability through replication; pre-registration prevents p-hacking and selective reporting—practices that have plagued other forensic domains.

\subsection{Requirement 6: Comprehensibility (AI Act)}

\textbf{Legal Language}: ``Accessible and comprehensible information'' (Article 13(3)(b)(ii))

\textbf{Technical Translation}: Target users (forensic analysts, judges, defendants) can correctly interpret what the explanation communicates, including its limitations.

\textbf{Validation Method}: User study with representative target audience. Present explanations and assess interpretation accuracy—do users correctly understand what is being communicated?

\textbf{Minimal Threshold}: $\geq 75\%$ correct interpretation

\textbf{Rationale}: Exceeds random chance for most interpretation tasks (typically $\geq 3$ options). Balances accessibility with technical accuracy—perfect comprehension may require simplification that sacrifices faithfulness.

Note: Comprehensibility is secondary to technical faithfulness. An explanation that is comprehensible but unfaithful violates GDPR/AI Act requirements.

\subsection{Requirement 7: Human Oversight (AI Act)}

\textbf{Legal Language}: Enable humans to ``make informed decisions'' and identify ``risks, anomalies, and signs of performance issues'' (Article 14)

\textbf{Technical Translation}: Operators receive per-instance reliability indicators that enable discrimination between reliable and unreliable explanations for specific cases.

\textbf{Validation Method}: Calibration study. For each explanation, provide confidence/quality score. On held-out validation set, measure whether these scores correlate with actual explanation accuracy. Compute AUC (area under ROC curve) for discriminating between reliable and unreliable explanations.

\textbf{Minimal Threshold}: AUC $\geq 0.75$ (Area Under the Receiver Operating Characteristic Curve, ranging 0.5--1.0, where 0.75 indicates the method correctly distinguishes reliable from unreliable explanations 75\% of the time)

\textbf{Rationale}: We adopt AUC $\geq 0.75$ from clinical prediction model validation (e.g., medical risk scores)~\cite{cohen1988statistical}, which shares forensic science's emphasis on consequential decision support with known error tolerance. Below 0.75, operators cannot meaningfully distinguish reliable from unreliable cases—oversight becomes pro forma rather than substantive.

\subsection{Summary}

Table~\ref{tab:minimal-evidence} summarizes minimal compliance requirements. Failure to meet any threshold indicates the system cannot demonstrate compliance with that requirement. Meeting all thresholds constitutes minimal evidence for responsible deployment—not a guarantee of perfection, but a baseline of scientific rigor analogous to standards in other forensic domains.

% Section 5: Forensic Reporting Template
% Humanized: Practitioner-ready, concrete examples, honest assessment

\section{Forensic Reporting Template}
\label{sec:template}

\subsection{Template Structure and Purpose}

To meet Daubert admissibility standards (Section~\ref{sec:background}) and regulatory transparency requirements, attribution validation results must be reported using a standardized structure. We provide a seven-field template designed for forensic analysts, legal professionals, and AI auditors. Each field addresses specific evidentiary criteria:

\begin{itemize}
\item \textbf{Field 1 (Method ID):} Addresses Daubert prong 2 (peer review)—specifies exact methods for reproducibility
\item \textbf{Field 2 (Parameters):} Operationalizes EU AI Act Article~15 (technical documentation)
\item \textbf{Field 3 ($\Delta$-Prediction Accuracy):} Addresses Daubert prong 1 (testability) and AI Act Article~13(3)(d) (accuracy metrics)
\item \textbf{Field 4 (CI Calibration):} Enables GDPR Article~22 contestation through uncertainty quantification
\item \textbf{Field 5 (Error Rates):} Addresses Daubert prong 3 (known error rates) and NRC 2009 forensic standards
\item \textbf{Field 6 (Limitations):} Ensures AI Act Article~13 transparency and prevents overclaiming
\item \textbf{Field 7 (Recommendation):} Provides actionable deployment guidance with explicit restrictions
\end{itemize}

Table~\ref{tab:template_fields} summarizes required information for each field.

\begin{table*}[!t]
\centering
\caption{Forensic Reporting Template: Required Information by Field}
\label{tab:template_fields}
\small
\begin{tabular}{@{}p{0.12\textwidth}p{0.35\textwidth}p{0.45\textwidth}@{}}
\toprule
\textbf{Field} & \textbf{Required Information} & \textbf{Purpose / Evidentiary Standard} \\ \midrule
1: Method ID & Attribution method (name, version, implementation), model architecture (training data, source, checkpoint) & Daubert prong 2 (peer review), reproducibility \\ \midrule
2: Parameters & Feature thresholds, counterfactual settings, statistical test parameters, pre-registered thresholds (with timestamp), dataset details & EU AI Act Art.~15 (technical documentation), transparency \\ \midrule
3: $\Delta$-Accuracy & Pearson $\rho$ (95\% CI, p-value), R$^2$, MAE/RMSE, scatter plot & Daubert prong 1 (testability), AI Act Art.~13(3)(d) (accuracy) \\ \midrule
4: CI Calibration & Empirical coverage rate (90\% CIs), binomial test p-value, stratified coverage (by score range), calibration plot & GDPR Art.~22 (contestation), uncertainty quantification \\ \midrule
5: Error Rates & Overall falsification rate (95\% CI), failure mode breakdown, demographic stratification (age, gender, skin tone), imaging condition stratification, known failure scenarios & Daubert prong 3 (error rates), NRC 2009 (objective standards) \\ \midrule
6: Limitations & Dataset limitations, model constraints, plausibility assumptions, demographic biases, out-of-scope scenarios & AI Act Art.~13 (transparency), prevent overclaiming \\ \midrule
7: Recommendation & Overall verdict (NOT FALSIFIED / FALSIFIED), confidence level (High / Moderate / Low), deployment recommendation (APPROVED / APPROVED with RESTRICTIONS / NOT APPROVED), specific restrictions, justification & Actionable guidance, explicit deployment criteria \\ \bottomrule
\end{tabular}
\end{table*}

\subsection{Field-by-Field Guidance}

We now detail each field with concrete examples. For brevity, we present one hypothetical completed report (Section~\ref{sec:example_report}) demonstrating moderate performance with demographic disparities—the most common real-world scenario.

\subsubsection{Field 1: Method Identification}

\textbf{Example:}

\texttt{METHOD IDENTIFICATION}

\textit{Attribution Method:} Gradient-Weighted Class Activation Mapping (Grad-CAM)~\cite{selvaraju2017grad}
\begin{itemize}
\item Implementation: Captum v0.6.0 (PyTorch 2.0.1)
\item Target layer: \texttt{conv5\_3} (final convolutional layer)
\item Output: 7$\times$7 spatial attribution map (49 features)
\item No modifications to standard implementation
\end{itemize}

\textit{Face Verification Model:} ArcFace ResNet-100~\cite{deng2019arcface}
\begin{itemize}
\item Architecture: ResNet-100 backbone, 512-D fully connected layer
\item Embeddings: L2-normalized (unit hypersphere $\mathbb{S}^{511}$)
\item Loss: Additive angular margin (m=0.5, s=64)
\item Training: VGGFace2-HQ (3.31M images, 9,131 identities)
\item Source: Official release, \texttt{github.com/deepinsight/insightface}
\item Checkpoint: \texttt{glint360k\_r100.pth}
\end{itemize}

This level of detail enables reproducibility. An independent auditor can obtain the exact model and implementation, re-run validation, and verify results.

\subsubsection{Field 2: Parameter Disclosure}

\textbf{Example (abbreviated):}

\texttt{PARAMETER DISCLOSURE}

\textit{Feature Thresholds:} $\theta_{\text{high}} = 0.7$, $\theta_{\text{low}} = 0.4$ (source: calibration set, N=500 LFW images, identities 0001--0500, no overlap with test set)

\textit{Counterfactual Settings:} $\delta_{\text{target}} = 0.8$~rad, K=200, T=100, $\alpha=0.01$, $\lambda=0.1$

\textit{Pre-Registered Thresholds:} $\tau_{\text{high}} = 0.75$~rad, $\tau_{\text{low}} = 0.55$~rad, $\epsilon = 0.15$~rad, $\rho_{\text{min}} = 0.7$, coverage 90--100\%, pre-registration timestamp: 2024-10-15, OSF ID: \texttt{[TO BE INSERTED]}

\textit{Dataset:} LFW test set, 1,000 image pairs (500 genuine, 500 impostor), demographics: 77\% male, 83\% light skin (based on available annotations)

Transparency is critical. Parameters must be disclosed even if they seem mundane (learning rates, sample sizes). Forensic scrutiny demands completeness.

\subsubsection{Field 3: $\Delta$-Prediction Accuracy}

\textbf{Example:}

\texttt{$\Delta$-PREDICTION ACCURACY}

\textit{Correlation:} Pearson $\rho = 0.73$ (95\% CI: [0.68, 0.78])

\textit{Hypothesis Test:} $H_0: \rho \leq 0.7$ vs.\ $H_1: \rho > 0.7$, p~=~0.012 $\Rightarrow$ \textbf{Reject $H_0$ at $\alpha=0.05$; primary endpoint MET}

\textit{Effect Size:} R$^2 = 0.53$ (53\% explained variance)—moderate predictive accuracy per Cohen (1988)

\textit{Prediction Error:} MAE~=~0.11~rad (6.3$^\circ$), RMSE~=~0.15~rad (8.6$^\circ$)

\textit{Interpretation:} Predicted geodesic distance changes demonstrate moderate-to-strong correlation with observed changes. Attributions show directional correctness (high-attribution features cause larger shifts) but imperfect magnitude estimation. For forensic purposes, this indicates attributions can distinguish important from unimportant features but should be interpreted cautiously for precise quantitative claims.

This interpretation acknowledges both strengths (correlation above threshold) and limitations (53\% explained variance leaves 47\% unexplained). Honest assessment builds trust with legal professionals who will rely on these reports.

\subsubsection{Field 4: Confidence Interval Calibration}

\textbf{Example (abbreviated):}

\texttt{CI CALIBRATION}

\textit{Coverage Rate:} 91.3\% (913 of 1,000 within predicted 90\% CI)

\textit{Calibration Test:} Binomial p~=~0.42 (fail to reject $H_0: p_{\text{coverage}} = 0.90$) $\Rightarrow$ \textbf{well-calibrated}

\textit{Interpretation:} Confidence intervals are reliable. Observed coverage (91.3\%) closely matches nominal 90\%. Practitioners can trust that reported CIs will contain true values $\sim$90\% of the time. Slight over-coverage (91.3\% vs.\ 90\%) suggests conservative (wider) intervals—acceptable in forensic contexts.

\subsubsection{Field 5: Known Error Rates and Failure Modes}

This is often the most important field for legal professionals. It directly addresses Daubert's error rate requirement.

\textbf{Example:}

\texttt{KNOWN ERROR RATES}

\textit{Overall Falsification Rate:} 38\% (380 of 1,000 test cases FALSIFIED), 95\% CI: [35.1\%, 40.9\%]

\textit{Failure Modes:}
\begin{itemize}
\item Non-Triviality: 2.1\% (21 cases)
\item Insufficient Statistical Evidence: 35.9\% (359 cases)
\item Separation Margin: 0\% (by design)
\end{itemize}

\textit{Demographic Stratification:}

\begin{table}[H]
\centering
\small
\begin{tabular}{lcc}
\toprule
\textbf{Group} & \textbf{N} & \textbf{Falsif. Rate} \\ \midrule
\multicolumn{3}{l}{\textit{Age}} \\
~~Young ($<$30y) & 287 & 34\% \\
~~Middle (30--50y) & 485 & 37\% \\
~~Older ($>$50y) & 228 & 45\% $\dagger$ \\ \midrule
\multicolumn{3}{l}{\textit{Gender}} \\
~~Male & 768 & 36\% \\
~~Female & 232 & 42\% \\ \midrule
\multicolumn{3}{l}{\textit{Skin Tone}} \\
~~Light & 831 & 35\% \\
~~Dark & 169 & 43\% \\ \bottomrule
\multicolumn{3}{l}{\footnotesize $\dagger$ HIGH DISPARITY: 11pp gap (older vs.\ young)}
\end{tabular}
\end{table}

\textit{Known Failure Scenarios:}
\begin{enumerate}
\item Extreme poses ($>$45$^\circ$ rotation): 52\% falsification rate
\item Heavy occlusion (surgical masks, hands covering face): 61\%
\item Low resolution ($<$80$\times$80 pixels): 48\%
\item Older individuals ($>$50 years): 45\% (age bias)
\end{enumerate}

\textit{Interpretation:} Method achieves NOT FALSIFIED status for 62\% of cases but exhibits systematic biases. Higher failure rates for older individuals, females, and darker skin tones indicate demographic disparities. Use with caution in forensically diverse contexts; restrict to high-quality frontal images; require mandatory demographic audit.

This honest reporting of failures builds credibility. Legal professionals can assess whether the method is appropriate for their specific case demographics.

\subsubsection{Field 6: Limitations and Scope}

\textbf{Example (abbreviated):}

\texttt{LIMITATIONS}

\textit{Dataset:} Validated on LFW (celebrity images, frontal poses, high resolution). May NOT generalize to surveillance footage, infrared imagery, or non-Western demographics.

\textit{Model:} ArcFace ResNet-100 specific. Results may differ for CosFace, transformer models, or different embedding dimensions.

\textit{Out-of-Scope:} Video, 3D faces, face identification (1:N search), adversarial robustness, real-time deployment ($\sim$4--9 seconds per image prohibits real-time use).

These limitations aren't weaknesses to hide—they define the scope within which claims hold. Transparent acknowledgment prevents misuse.

\subsubsection{Field 7: Recommendation and Confidence Assessment}

This field translates technical findings into actionable guidance.

\textbf{Example (Moderate Performance with Restrictions):}

\texttt{RECOMMENDATION}

\textit{Verdict:} \textbf{NOT FALSIFIED}

\textit{Confidence Level:} \textbf{MODERATE} (correlation $\rho=0.73$ above threshold, but 38\% falsification rate and demographic disparities)

\textit{Deployment Recommendation:} \textbf{APPROVED for forensic use with RESTRICTIONS}

\textit{Mandatory Restrictions:}
\begin{enumerate}
\item Image quality: Minimum 100$\times$100 pixels, pose $<$30$^\circ$ rotation, no heavy occlusion
\item Demographic audit: Report stratified performance for each case's demographic category
\item Human expert review: Required when attributions highlight unusual regions (e.g., $>$30\% importance on background)
\item Uncertainty disclosure: Always report 90\% confidence intervals
\item Evidentiary limitation: Use as investigative aid, NOT sole evidence; require corroboration
\end{enumerate}

\textit{Contraindications (DO NOT USE):}
\begin{itemize}
\item Surveillance footage $<$80$\times$80 pixels
\item Profile views ($>$30$^\circ$ rotation)
\item Video-based verification
\item Real-time deployment
\end{itemize}

\textit{Justification:} Moderate predictive accuracy ($\rho=0.73$, R$^2=0.53$) and well-calibrated uncertainty (91.3\% coverage) indicate attributions provide useful forensic insights. However, 38\% falsification rate and demographic disparities (11pp gap for age) necessitate restrictions. These balance utility (enabling use where validation is strongest) with safety (preventing misuse in scenarios where validation fails).

\subsection{Example Completed Report}
\label{sec:example_report}

Due to space constraints, we present an abbreviated complete report demonstrating the template in practice. Full examples with all fields are available in supplementary materials.

\begin{table}[H]
\centering
\scriptsize
\caption{Hypothetical Forensic Attribution Validation Report (Abbreviated)}
\begin{tabular}{@{}p{0.95\linewidth}@{}}
\toprule
\textbf{FORENSIC ATTRIBUTION VALIDATION REPORT} \\
\textit{Case ID:} [Redacted] \quad \textit{Date:} 2024-10-20 \quad \textit{Analyst:} [Name, Credentials] \\ \midrule
\textbf{Field 1: Method ID} \\
Grad-CAM (Captum v0.6.0) | ArcFace ResNet-100 (VGGFace2-HQ, official release) \\ \midrule
\textbf{Field 2: Parameters} \\
$\theta_{\text{high}}=0.7$, $\theta_{\text{low}}=0.4$ | $\delta_{\text{target}}=0.8$~rad, K=200 | Pre-reg: OSF \texttt{[ID]}, 2024-10-15 | LFW, N=1,000 \\ \midrule
\textbf{Field 3: $\Delta$-Accuracy} \\
$\rho=0.73$ [0.68, 0.78], p=0.012 (\textbf{MET}) | R$^2$=0.53 | MAE=0.11~rad (6.3$^\circ$) \\ \midrule
\textbf{Field 4: CI Calibration} \\
Coverage: 91.3\%, binomial p=0.42 (\textbf{well-calibrated}) \\ \midrule
\textbf{Field 5: Error Rates} \\
Falsif.: 38\% [35.1\%, 40.9\%] | Age: 34\%/37\%/45\% (young/mid/older, \textbf{11pp disparity}) | Failure: pose$>$30$^\circ$ (52\%), occlusion (61\%) \\ \midrule
\textbf{Field 6: Limitations} \\
LFW (celebrity, frontal, high-res) | ArcFace-specific | Out-of-scope: video, 3D, real-time \\ \midrule
\textbf{Field 7: Recommendation} \\
Verdict: \textbf{NOT FALSIFIED} | Confidence: \textbf{MODERATE} | Deployment: \textbf{APPROVED with RESTRICTIONS} \\
Restrictions: (1) Image quality $\geq$100px, pose$<$30$^\circ$; (2) Demographic audit; (3) Expert review for unusual attributions; (4) Report 90\% CIs; (5) Investigative aid only, require corroboration \\
Contraindications: Surveillance$<$80px, profile views, video, real-time \\ \midrule
\textit{Analyst Signature:} [Signature] \quad \textit{Supervisor:} [Signature] \quad \textit{Date:} 2024-10-20 \\ \bottomrule
\end{tabular}
\end{table}

This condensed format is suitable for case files. The full report (with visualizations, detailed tables, statistical test outputs) would run 10--15 pages.

\subsection{Usage Guidance for Practitioners}

\textbf{When to complete this template:}
\begin{enumerate}
\item Before deploying a new attribution method in forensic investigations
\item After model updates (retraining, fine-tuning)
\item When dataset shifts significantly (e.g., surveillance footage after validating on LFW)
\item Annual review (periodic revalidation)
\end{enumerate}

\textbf{Legal and ethical considerations:}
\begin{itemize}
\item \textit{Daubert compliance:} Field~1 (peer review), Field~3 (testability), Field~5 (error rates)
\item \textit{GDPR/AI Act:} Field~2 (logic), Field~3 (accuracy), Field~6 (transparency)
\item \textit{Transparency:} Always disclose template to defendants, legal counsel, oversight bodies
\item \textit{Disclosure:} Make available upon FOIA/public records requests; include as exhibit in court
\end{itemize}

\textbf{Template versioning:} This is version 1.0 (2024). As regulatory frameworks evolve or new scientific evidence emerges, we will update field requirements. Check \texttt{[repository URL]} for the latest version.

\section{Discussion and Policy Implications}

The analysis reveals a fundamental mismatch between regulatory intent and technical practice. Legal frameworks mandate explainability for high-stakes AI systems, yet current XAI deployment lacks the validation foundations that regulators, courts, and practitioners require. This section discusses implications for key stakeholders and proposes actionable recommendations.

\subsection{For Regulators and Standards Bodies}

\subsubsection{Gap Identified}

Regulatory language—GDPR's ``meaningful information,'' the AI Act's ``appropriate accuracy''—lacks technical operationalization. This ambiguity enables ``checkbox compliance'' where systems generate explanations without validating their quality.

\subsubsection{Recommendations}

\textbf{Establish Technical Standards}. Regulatory bodies (e.g., EU AI Office, NIST) should publish technical standards specifying minimal evidence requirements for XAI validation, analogous to existing standards for DNA analysis or digital forensics. The seven requirements and thresholds proposed here provide a starting point.

\textbf{Mandate Validation Protocols}. Require pre-registered validation protocols with published benchmarks before high-risk AI systems can be deployed. Protocols should specify acceptance thresholds before data collection to prevent post-hoc p-hacking.

\textbf{Require Error Rate Disclosure}. Mandate that deployed systems document known failure modes and conditional error rates (e.g., explanation faithfulness by demographic group, image quality, score range). This mirrors Daubert's error rate requirement and enables risk-informed deployment.

\textbf{Periodic Revalidation}. Establish timelines for revalidation (e.g., annually) as models, methods, and datasets evolve. Face recognition systems aren't static—validation cannot be one-time certification.

\textbf{Demographic Fairness Requirements}. Extend validation requirements to include fairness thresholds—explanation faithfulness must meet minimal thresholds for all demographic groups, not just aggregate populations. As illustrated in our Grad-CAM example (Section 5.2), faithfulness for dark-skinned females falls to $\rho = 0.64$ compared to $\rho = 0.68$ overall—a disparity that exacerbates existing bias concerns in face recognition systems~\cite{grother2019frvt}.

\textbf{Precedent}: The European Union's Medical Device Regulation (MDR 2017/745)~\cite{mdr2017} provides a model for risk-based AI oversight with technical standards, conformity assessment, and post-market surveillance. Adapting these principles to AI explainability could establish rigorous governance.

\subsection{For System Developers and Vendors}

\subsubsection{Gap Identified}

Current development practices treat explanation generation as a feature add-on, not a core system requirement with validation obligations.

\subsubsection{Recommendations}

\textbf{Validation-First Development}. Incorporate faithfulness validation into the development lifecycle from the start, not as a post-deployment afterthought. XAI methods should be selected based on empirical validation performance, not popularity or visual appeal.

\textbf{Benchmark Participation}. Contribute to community development of standardized XAI benchmarks with ground truth. Publish validation results to establish credibility and enable comparative evaluation.

\textbf{Uncertainty Quantification}. Provide calibrated confidence intervals for explanations using conformal prediction~\cite{vovk2005conformal} or Bayesian methods. Operators need uncertainty estimates to calibrate trust appropriately.

\textbf{Per-Instance Quality Scores}. Develop and deploy reliability indicators that enable operators to identify when specific explanations are unreliable. Aggregate validation metrics are insufficient—operators need case-level guidance. The Grad-CAM example's AUC of 0.71 shows this remains challenging but achievable.

\textbf{Transparent Limitation Documentation}. Clearly communicate known failure modes in user documentation and system interfaces. When an image falls into a documented failure mode (e.g., profile face, low resolution), flag this for operators automatically.

\textbf{Open Validation}. Publish validation protocols and results in peer-reviewed venues. Proprietary systems can be validated on public benchmarks without disclosing model weights.

\textbf{Business Case}: While validation adds development costs, it mitigates liability risks. Systems that contribute to wrongful arrests or fail Daubert challenges expose vendors to lawsuits. Proactive validation provides defensible due diligence.

\subsection{For Auditors and Oversight Bodies}

\subsubsection{Gap Identified}

Auditors tasked with assessing AI system compliance lack technical tools and standards to evaluate explanation quality.

\subsubsection{Recommendations}

\textbf{Adopt Standardized Evaluation Protocols}. Use the compliance template (Section 5) or similar structured frameworks to systematically assess XAI validation. Require vendors to provide completed templates as part of procurement or compliance review.

\textbf{Independent Validation}. Don't rely solely on vendor-provided validation studies. Conduct independent testing on held-out datasets, particularly for high-stakes deployments. The difference between vendor claims and independent assessment can be substantial.

\textbf{Red Team Testing}. Employ adversarial evaluation to identify conditions under which explanations fail. Test edge cases: demographic groups underrepresented in training data, challenging poses, adversarial perturbations. The five failure modes identified in the Grad-CAM example emerged precisely from this kind of stratified analysis.

\textbf{Ongoing Monitoring}. Compliance isn't binary or static. Establish continuous monitoring programs that track explanation quality metrics over time as systems evolve and operational conditions change.

\textbf{Transparency Requirements}. Require that systems undergoing audit provide sufficient access for replication—validation datasets, model APIs (even if weights remain proprietary), and detailed methodology documentation.

\textbf{Precedent}: Financial services auditing (e.g., SOX compliance for algorithmic trading) provides models for independent technical evaluation of complex systems with legal accountability.

\subsection{For Courts and Legal Professionals}

\subsubsection{Gap Identified}

Judges and attorneys lack technical expertise to evaluate XAI evidence presented in criminal proceedings, leading to either uncritical acceptance or blanket exclusion.

\subsubsection{Recommendations}

\textbf{Daubert Challenges for XAI Evidence}. When face recognition explanations are introduced as evidence, defense attorneys should challenge admissibility under Daubert. Critical questions include: Has the XAI method been validated with documented error rates across different conditions? Are published standards controlling its operation, or does deployment rely on ad-hoc vendor practices? Can the explanation be tested through controlled experiments that could potentially refute its claims? Whether peer review has occurred matters—but more importantly, whether that review addressed validation rather than just method description.

The Grad-CAM example's failure on Standards (no pre-registration), Meaningful Information ($\rho = 0.68$), and Accuracy (76\%) would provide grounds for challenge.

\textbf{Expert Witness Standards}. Courts should require that expert witnesses presenting XAI evidence have conducted (or reviewed) rigorous validation studies, not merely familiarity with the XAI tool. An expert testifying ``we used Grad-CAM'' without validation data should face cross-examination on faithfulness, error rates, and failure modes.

\textbf{Judicial Education}. Provide training for judges on XAI fundamentals and validation principles through judicial education programs (e.g., Federal Judicial Center). Enable informed gatekeeping without requiring deep technical expertise. Simple questions—``What is the correlation between predicted and actual score changes?'' ``What percentage of cases fall into known failure modes?''—can reveal whether proper validation occurred.

\textbf{Standard Jury Instructions}. Develop model jury instructions for cases involving face recognition evidence. Jurors need to understand the critical distinction between model accuracy (how often the system correctly matches faces) and explanation accuracy (whether the highlighted features actually drove the decision). Validation metrics like correlation coefficients and error rates should be explained in accessible terms. Known limitations—such as degraded performance on profile faces or low-resolution images—deserve explicit mention, as does guidance on the appropriate weight XAI evidence should carry relative to other forensic methods.

\textbf{Precedent Development}. As cases involving XAI evidence accumulate, appellate decisions should establish precedent on admissibility standards, clarifying how Daubert applies to explainability methods specifically.

Key point: technical faithfulness is necessary but not sufficient for legal admissibility. Even validated explanations must be probative, reliable in the specific case context, and not unduly prejudicial.

\subsection{Who Benefits From Validated XAI?}

The proposed validation framework serves multiple stakeholders with aligned interests in accuracy and accountability:

\textbf{Defendants and Accused Persons}. Validated explanations enable effective challenges to face recognition evidence. If an explanation fails validation thresholds, defense attorneys have grounds to argue for exclusion or reduced evidentiary weight.

\textbf{Law Enforcement and Forensic Analysts}. Validated systems protect agencies from liability risks associated with wrongful arrests. Knowing when explanations are reliable versus unreliable enables more effective investigations and resource allocation. The 23\% rejection rate in the Grad-CAM example means analysts can focus expert review where it's most needed.

\textbf{Regulatory Agencies}. Validated systems provide clear compliance evidence, reducing enforcement ambiguity and enabling risk-based oversight prioritization.

\textbf{System Developers}. Validation standards create level playing fields and enable differentiation based on empirical performance rather than marketing claims.

\textbf{Judges and Courts}. Validated evidence reduces Daubert hearing complexity and provides clear admissibility criteria, streamlining proceedings.

\textbf{Society}. Reduced wrongful identifications protect civil liberties; transparent accountability mechanisms build public trust in beneficial uses of face recognition technology.

\subsection{Remaining Gaps and Future Directions}

While this article provides a framework for operationalizing existing regulatory requirements, several gaps require ongoing attention:

\textbf{Threshold Consensus}. The proposed thresholds ($\rho \geq 0.70$, 80\% accuracy, etc.) are informed by statistical practice and analogous domains but require community consensus through standards development processes (ISO, NIST, professional societies).

\textbf{Dynamic Adaptation}. Validation standards must evolve as XAI methods, face verification architectures, and adversarial threats develop. Static standards risk obsolescence.

\textbf{Cross-Jurisdictional Harmonization}. U.S., EU, and other jurisdictions have different legal frameworks. International standards harmonization could reduce compliance complexity for multinational deployments.

\textbf{Fairness Integration}. Current regulatory frameworks address explainability and accuracy but lack explicit fairness requirements. Future standards should mandate that validation thresholds are met across demographic groups. Our Grad-CAM example revealed this gap concretely: while aggregate faithfulness reached $\rho = 0.68$, dark-skinned females experienced $\rho = 0.64$—falling further below the 0.70 threshold and compounding algorithmic bias concerns.

\textbf{Alternative Explanation Paradigms}. This article focuses on attribution-based XAI (saliency maps). Other paradigms—example-based explanations, concept-based interpretability, natural language rationales—require separate validation frameworks.

\subsection{A Call for Evidence-Based Policy}

Current XAI practice has operated in a normative vacuum—researchers develop methods based on intuition, vendors deploy based on demand, and regulators mandate explainability without technical specificity. This article proposes a shift toward \textbf{evidence-based explainability policy}:

\begin{itemize}[itemsep=2pt]
    \item Requirements grounded in measurable criteria
    \item Validation following scientific method principles
    \item Standards informed by empirical performance data
    \item Ongoing evaluation as systems and threats evolve
\end{itemize}

This mirrors the evolution of other forensic domains. DNA analysis, fingerprint comparison, and ballistic matching once lacked rigorous scientific foundations. Following high-profile wrongful convictions and critical reports (e.g., the 2009 NRC report on forensic science~\cite{nrc2009}), these fields developed validation protocols, error rate disclosure requirements, and proficiency testing standards.

Face recognition XAI stands at a similar inflection point. Documented wrongful arrests~\cite{hill2020williams,hill2023oliver} and regulatory mandates~\cite{euaiact2024,gdpr2016,daubert1993} create urgency for evidence-based standards. The framework proposed here—seven evidentiary requirements with operationalized thresholds and a compliance template—provides a starting point, not a final answer. Refinement through multi-stakeholder collaboration (researchers, practitioners, regulators, civil liberties advocates) is essential.

But the status quo—deploying explanations without validation—is scientifically indefensible and legally untenable. The time for evidence-based explainability policy is now.

\section{Conclusion}

Face recognition systems deployed in law enforcement operate at the intersection of impressive technical capabilities and profound accountability challenges. These systems achieve high matching accuracy—often exceeding 99.7\% on benchmark datasets—yet their decision-making processes remain opaque. Explainable AI methods offer a path toward transparency by generating visual attributions highlighting influential facial features. However, current practice exhibits a critical gap: explanations are generated without rigorous validation of their faithfulness to model reasoning.

This gap matters because regulatory frameworks increasingly mandate not just explanations, but \emph{accurate} explanations. The EU AI Act requires ``accurate, accessible, and comprehensible information''; GDPR demands ``meaningful information about the logic involved''; and U.S. courts applying Daubert standards require testable methods with known error rates. Current XAI practice—producing explanations without validating them—cannot demonstrate compliance with these requirements.

Through systematic analysis of three major regulatory frameworks (EU AI Act, GDPR, Daubert standard), we identified seven core evidentiary requirements: meaningful information, testability, known error rates, appropriate accuracy, adherence to standards, comprehensibility, and human oversight support. For each requirement, we proposed minimal technical evidence specifications, validation methods, and acceptance thresholds that operationalize vague legal concepts into measurable criteria.

The proposed framework reveals a troubling pattern: current practice achieves compliance in form but not substance. Systems generate explanations (satisfying literal regulatory language) without validation (failing the policy intent). This form-versus-substance gap exposes legal systems, defendants, and agencies to serious risks—wrongful identifications based on misleading explanations, Daubert inadmissibility challenges that derail prosecutions, and regulatory enforcement uncertainty that creates legal exposure for deploying agencies.

The path forward requires evidence-based policy. We conclude with concrete recommendations:

\textbf{For Regulators}: Establish technical standards operationalizing vague legal language (``meaningful information,'' ``appropriate accuracy'') into measurable criteria. Mandate pre-registered validation protocols with published benchmarks. Require error rate disclosure including demographic stratification. Establish periodic revalidation requirements as systems evolve.

\textbf{For Developers}: Adopt validation-first development practices where XAI methods are selected based on empirical performance, not popularity. Contribute to community benchmark development. Provide calibrated uncertainty estimates and per-instance quality scores. Transparently document limitations and known failure modes. Publish validation protocols and results in peer-reviewed venues.

\textbf{For Auditors}: Adopt standardized evaluation protocols like the compliance template proposed here. Conduct independent validation beyond vendor claims. Employ red team testing for edge cases and demographic subgroups. Establish continuous monitoring programs. Require transparency enabling replication.

\textbf{For Courts}: Subject XAI evidence to rigorous Daubert scrutiny—testability, error rates, published standards, peer review. Require expert witnesses to demonstrate validation, not merely familiarity with tools. Develop judicial education programs on XAI validation principles. Create standard jury instructions explaining the distinction between model accuracy and explanation accuracy.

The compliance template and example validation (Grad-CAM on ArcFace) demonstrate both the feasibility and necessity of systematic assessment. The example system passed 3/7 requirements—sufficient for investigative leads under supervision, but not for primary evidence in legal proceedings. This nuanced assessment, grounded in measurable criteria, enables risk-informed deployment decisions that balance innovation with accountability.

Face recognition XAI stands where DNA analysis stood decades ago—at an inflection point between ad-hoc practice and scientific rigor. DNA analysis evolved from a novel forensic tool with uncertain reliability into a cornerstone of criminal justice, but only after developing validation protocols, error rate disclosure requirements, and proficiency testing standards. This evolution followed high-profile wrongful convictions and critical National Research Council reports demanding scientific foundations for forensic methods.

Face recognition XAI faces similar pressures. Robert Williams and Michael Oliver—arrested based on false matches—represent visible failures of a broader accountability gap. The EU AI Act, GDPR Article 22, and Daubert standards create legal obligations. Yet between regulatory mandate and technical reality lies a translation problem: how to operationalize legal requirements into measurable technical criteria?

This article provides that translation. The seven evidentiary requirements, minimal thresholds, and compliance template transform abstract legal concepts into concrete technical specifications. These specifications aren't final answers—threshold values require community consensus through standards development processes, and validation methods will evolve as XAI techniques advance. But they provide a starting point grounded in statistical practice, forensic science precedent, and existing regulatory frameworks.

The status quo—deploying explanations without validation—is scientifically indefensible and legally untenable. Explanations that are 68\% faithful (below our 0.70 threshold) and 76\% accurate (below our 80\% threshold) may be better than nothing, but they're insufficient for contexts where liberty is at stake. The documented failure modes—profile faces, low resolution, occlusion, demographic disparities, borderline scores—reveal systematic patterns that operators must understand to make informed decisions.

The framework proposed here serves multiple stakeholders with aligned interests in accuracy and accountability. Defendants gain tools to challenge unreliable evidence. Law enforcement agencies protect themselves from liability while improving investigative effectiveness. Regulators obtain clear compliance criteria. Developers establish level playing fields. Courts streamline admissibility determinations. Society benefits from reduced wrongful identifications and increased trust in legitimate applications of face recognition technology.

Achieving these benefits requires multi-stakeholder collaboration—which is both urgent and achievable. Standards bodies like ISO and NIST must convene researchers, practitioners, regulators, and civil liberties advocates to refine thresholds and develop consensus benchmarks. We observe that research communities are beginning to shift from prioritizing subjective interpretability to objective faithfulness validation, though publication incentives still favor novel methods over rigorous evaluation. Vendors face perhaps the hardest challenge: embracing validation-first development even when it reveals uncomfortable limitations about their products. Courts, finally, need expertise to evaluate technical evidence rigorously—moving beyond uncritical deference to expert testimony toward informed gatekeeping.

This collaboration is both urgent and achievable. The technical foundations exist—counterfactual validation, conformal prediction, ground truth benchmarks. The legal frameworks create obligation and motivation. The wrongful arrest cases demonstrate tangible harms of inaction. What remains is translation and implementation.

This article provides the translation: seven requirements, minimal thresholds, a compliance template, and stakeholder-specific recommendations. Implementation requires commitment from regulators, developers, auditors, and courts to establish evidence-based validation as the norm rather than the exception.

The choice is clear. We can continue deploying explanations without validation, hoping that systems are trustworthy while lacking tools to verify that trust. Or we can demand evidence—testable predictions, known error rates, published standards, peer-reviewed protocols. The former preserves the status quo and its attendant risks. The latter builds accountability into AI systems from the foundation.

Face recognition technology offers genuine benefits for public safety and security. But those benefits cannot come at the cost of civil liberties or scientific integrity. Validated explainability bridges that gap—enabling beneficial applications while providing the accountability mechanisms that protect individual rights. The framework proposed here shows the path. Now comes the hard work of walking it.


% Tables
\newpage
\section*{Tables}

\begin{table}[h]
\centering
\caption{Regulatory Requirements vs. Current XAI Practice}
\label{tab:requirements-gap}
\footnotesize
\begin{tabular}{@{}p{3.5cm}p{3cm}p{2.5cm}p{4cm}@{}}
\toprule
\textbf{Requirement} & \textbf{Current Practice} & \textbf{Gap} & \textbf{Impact} \\
\midrule
\textbf{Meaningful Information} (GDPR Art. 22) &
Visual saliency maps (Grad-CAM, SHAP) produced without validation &
No verification that highlighted regions actually influenced decision &
Individuals cannot effectively contest algorithmic decisions; explanations may be post-hoc rationalizations \\
\addlinespace
\textbf{Appropriate Transparency \& Accuracy} (AI Act Art. 13) &
Documentation describes XAI method used &
No evidence that explanations are \emph{accurate} representations of model reasoning &
Operators may misinterpret unreliable explanations, leading to incorrect override decisions \\
\addlinespace
\textbf{Testability} (Daubert) &
XAI methods produce outputs but lack falsifiable hypotheses &
Explanations cannot be empirically tested or refuted through controlled experiments &
Fails judicial admissibility standards in U.S. courts; cannot demonstrate scientific validity \\
\addlinespace
\textbf{Known Error Rates} (Daubert; AI Act Art. 14) &
Error rates reported for face verification accuracy, not for explanation faithfulness &
No quantified failure modes of attribution methods; investigators don't know when explanations are unreliable &
Cannot assess reliability of specific explanation; may trust misleading explanations in critical cases \\
\addlinespace
\textbf{Standards} (Daubert) &
Ad-hoc deployment of XAI methods without published protocols or acceptance thresholds &
No consensus standards for when explanation quality is sufficient for forensic use &
Inconsistent practices across agencies; no basis for inter-agency comparison or legal challenges \\
\addlinespace
\textbf{Appropriate Accuracy} (AI Act Art. 13) &
Verification models report accuracy metrics, but explanation accuracy is assumed, not measured &
Attribution methods may systematically misidentify important features (studies show 40--69\% accuracy) &
High model accuracy does not guarantee explanation reliability; false confidence in forensic applications \\
\addlinespace
\textbf{Human Oversight} (AI Act Art. 14) &
Operators review XAI outputs without tools to assess explanation quality &
Operators lack meta-information about explanation reliability for specific cases &
Cannot distinguish reliable from unreliable explanations; oversight becomes pro forma rather than substantive \\
\bottomrule
\end{tabular}
\end{table}

\newpage

\begin{landscape}
\begin{table}[h]
\centering
\caption{Minimal Evidence Requirements for XAI Compliance}
\label{tab:minimal-evidence}
\scriptsize
\begin{tabular}{@{}p{3cm}p{3.5cm}p{3cm}p{2.5cm}p{3cm}@{}}
\toprule
\textbf{Requirement} & \textbf{Minimal Technical Evidence} & \textbf{Validation Method} & \textbf{Acceptance Threshold} & \textbf{Reporting Format} \\
\midrule
\textbf{Meaningful Information} (GDPR Art. 22) &
Faithful attribution map where highlighted regions actually influence model decision &
Counterfactual score prediction: $\Delta s_{\text{predicted}}$ vs. $\Delta s_{\text{actual}}$ &
Pearson $\rho \geq 0.70$ between predicted and actual score changes &
``Attribution faithfulness: $\rho = X.XX$ [95\% CI: X.XX--X.XX]'' \\
\addlinespace
\textbf{Testability} (Daubert) &
Falsifiable hypothesis about feature importance that can be empirically tested &
Perturbation experiments with statistical hypothesis testing &
$p < 0.05$ for $H_0$: attribution is random guessing; Cohen's $d \geq 0.5$ &
``Testability: $\chi^2 = XX$, $p < 0.001$; attributions significantly predict score changes'' \\
\addlinespace
\textbf{Known Error Rates} (Daubert; AI Act Art. 14) &
(1) Confidence interval calibration for predictions; (2) Documented failure modes &
Conformal prediction for CI coverage; stratified evaluation by demographics/conditions &
(1) 90--95\% coverage for stated CIs; (2) Complete failure mode documentation &
``CI calibration: 92\% coverage at 90\% CI. Known failure modes: [list]. Rejection rate: X\%'' \\
\addlinespace
\textbf{Appropriate Accuracy} (AI Act Art. 13) &
Quantified explanation accuracy independent of model accuracy &
Ground truth test cases where true feature importance is known &
Explanation accuracy $\geq 80\%$ on ground truth benchmarks &
``Explanation accuracy: 85\% correct feature identification [benchmark: controlled perturbation suite]'' \\
\addlinespace
\textbf{Standards} (Daubert) &
Pre-registered validation protocol with published acceptance criteria &
Peer-reviewed validation study using standardized benchmark &
Methods published in peer-reviewed venue; benchmark publicly available &
``Validation protocol: [citation]. Benchmark: [name]. Results: [metrics]'' \\
\addlinespace
\textbf{Comprehensibility} (AI Act Art. 13) &
Explanation + uncertainty quantification + limitations documentation &
User study or expert evaluation of comprehensibility (secondary to technical faithfulness) &
Target audience can correctly interpret explanation's meaning and limitations $\geq 75\%$ of time &
``Comprehensibility: XX\% correct interpretation by [target audience] in controlled study'' \\
\addlinespace
\textbf{Human Oversight} (AI Act Art. 14) &
Meta-level reliability indicator for each explanation (per-instance quality score) &
Prediction confidence calibrated to actual accuracy on held-out validation set &
Operator can discriminate between reliable/unreliable explanations with AUC $\geq 0.75$ &
``Reliability indicator: AUC = 0.XX for predicting explanation error'' \\
\bottomrule
\end{tabular}
\end{table}
\end{landscape}


% Bibliography - IEEE style
\bibliographystyle{IEEEtran}
\bibliography{references}

\end{document}
