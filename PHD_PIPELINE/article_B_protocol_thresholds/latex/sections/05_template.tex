% Section 5: Forensic Reporting Template
% Humanized: Practitioner-ready, concrete examples, honest assessment

\section{Forensic Reporting Template}
\label{sec:template}

\subsection{Template Structure and Purpose}

To meet Daubert admissibility standards (Section~\ref{sec:background}) and regulatory transparency requirements, attribution validation results must be reported using a standardized structure. We provide a seven-field template designed for forensic analysts, legal professionals, and AI auditors. Each field addresses specific evidentiary criteria:

\begin{itemize}
\item \textbf{Field 1 (Method ID):} Addresses Daubert prong 2 (peer review)—specifies exact methods for reproducibility
\item \textbf{Field 2 (Parameters):} Operationalizes EU AI Act Article~15 (technical documentation)
\item \textbf{Field 3 ($\Delta$-Prediction Accuracy):} Addresses Daubert prong 1 (testability) and AI Act Article~13(3)(d) (accuracy metrics)
\item \textbf{Field 4 (CI Calibration):} Enables GDPR Article~22 contestation through uncertainty quantification
\item \textbf{Field 5 (Error Rates):} Addresses Daubert prong 3 (known error rates) and NRC 2009 forensic standards
\item \textbf{Field 6 (Limitations):} Ensures AI Act Article~13 transparency and prevents overclaiming
\item \textbf{Field 7 (Recommendation):} Provides actionable deployment guidance with explicit restrictions
\end{itemize}

Table~\ref{tab:template_fields} summarizes required information for each field.

\begin{table*}[!t]
\centering
\caption{Forensic Reporting Template: Required Information by Field}
\label{tab:template_fields}
\small
\begin{tabular}{@{}p{0.12\textwidth}p{0.35\textwidth}p{0.45\textwidth}@{}}
\toprule
\textbf{Field} & \textbf{Required Information} & \textbf{Purpose / Evidentiary Standard} \\ \midrule
1: Method ID & Attribution method (name, version, implementation), model architecture (training data, source, checkpoint) & Daubert prong 2 (peer review), reproducibility \\ \midrule
2: Parameters & Feature thresholds, counterfactual settings, statistical test parameters, pre-registered thresholds (with timestamp), dataset details & EU AI Act Art.~15 (technical documentation), transparency \\ \midrule
3: $\Delta$-Accuracy & Pearson $\rho$ (95\% CI, p-value), R$^2$, MAE/RMSE, scatter plot & Daubert prong 1 (testability), AI Act Art.~13(3)(d) (accuracy) \\ \midrule
4: CI Calibration & Empirical coverage rate (90\% CIs), binomial test p-value, stratified coverage (by score range), calibration plot & GDPR Art.~22 (contestation), uncertainty quantification \\ \midrule
5: Error Rates & Overall falsification rate (95\% CI), failure mode breakdown, demographic stratification (age, gender, skin tone), imaging condition stratification, known failure scenarios & Daubert prong 3 (error rates), NRC 2009 (objective standards) \\ \midrule
6: Limitations & Dataset limitations, model constraints, plausibility assumptions, demographic biases, out-of-scope scenarios & AI Act Art.~13 (transparency), prevent overclaiming \\ \midrule
7: Recommendation & Overall verdict (NOT FALSIFIED / FALSIFIED), confidence level (High / Moderate / Low), deployment recommendation (APPROVED / APPROVED with RESTRICTIONS / NOT APPROVED), specific restrictions, justification & Actionable guidance, explicit deployment criteria \\ \bottomrule
\end{tabular}
\end{table*}

\subsection{Field-by-Field Guidance}

We now detail each field with concrete examples. For brevity, we present one hypothetical completed report (Section~\ref{sec:example_report}) demonstrating moderate performance with demographic disparities—the most common real-world scenario.

\subsubsection{Field 1: Method Identification}

\textbf{Example:}

\texttt{METHOD IDENTIFICATION}

\textit{Attribution Method:} Gradient-Weighted Class Activation Mapping (Grad-CAM)~\cite{selvaraju2017grad}
\begin{itemize}
\item Implementation: Captum v0.6.0 (PyTorch 2.0.1)
\item Target layer: \texttt{conv5\_3} (final convolutional layer)
\item Output: 7$\times$7 spatial attribution map (49 features)
\item No modifications to standard implementation
\end{itemize}

\textit{Face Verification Model:} ArcFace ResNet-100~\cite{deng2019arcface}
\begin{itemize}
\item Architecture: ResNet-100 backbone, 512-D fully connected layer
\item Embeddings: L2-normalized (unit hypersphere $\mathbb{S}^{511}$)
\item Loss: Additive angular margin (m=0.5, s=64)
\item Training: VGGFace2-HQ (3.31M images, 9,131 identities)
\item Source: Official release, \texttt{github.com/deepinsight/insightface}
\item Checkpoint: \texttt{glint360k\_r100.pth}
\end{itemize}

This level of detail enables reproducibility. An independent auditor can obtain the exact model and implementation, re-run validation, and verify results.

\subsubsection{Field 2: Parameter Disclosure}

\textbf{Example (abbreviated):}

\texttt{PARAMETER DISCLOSURE}

\textit{Feature Thresholds:} $\theta_{\text{high}} = 0.7$, $\theta_{\text{low}} = 0.4$ (source: calibration set, N=500 LFW images, identities 0001--0500, no overlap with test set)

\textit{Counterfactual Settings:} $\delta_{\text{target}} = 0.8$~rad, K=200, T=100, $\alpha=0.01$, $\lambda=0.1$

\textit{Pre-Registered Thresholds:} $\tau_{\text{high}} = 0.75$~rad, $\tau_{\text{low}} = 0.55$~rad, $\epsilon = 0.15$~rad, $\rho_{\text{min}} = 0.7$, coverage 90--100\%, pre-registration timestamp: 2024-10-15, OSF ID: \texttt{[TO BE INSERTED]}

\textit{Dataset:} LFW test set, 1,000 image pairs (500 genuine, 500 impostor), demographics: 77\% male, 83\% light skin (based on available annotations)

Transparency is critical. Parameters must be disclosed even if they seem mundane (learning rates, sample sizes). Forensic scrutiny demands completeness.

\subsubsection{Field 3: $\Delta$-Prediction Accuracy}

\textbf{Example:}

\texttt{$\Delta$-PREDICTION ACCURACY}

\textit{Correlation:} Pearson $\rho = 0.73$ (95\% CI: [0.68, 0.78])

\textit{Hypothesis Test:} $H_0: \rho \leq 0.7$ vs.\ $H_1: \rho > 0.7$, p~=~0.012 $\Rightarrow$ \textbf{Reject $H_0$ at $\alpha=0.05$; primary endpoint MET}

\textit{Effect Size:} R$^2 = 0.53$ (53\% explained variance)—moderate predictive accuracy per Cohen (1988)

\textit{Prediction Error:} MAE~=~0.11~rad (6.3$^\circ$), RMSE~=~0.15~rad (8.6$^\circ$)

\textit{Interpretation:} Predicted geodesic distance changes demonstrate moderate-to-strong correlation with observed changes. Attributions show directional correctness (high-attribution features cause larger shifts) but imperfect magnitude estimation. For forensic purposes, this indicates attributions can distinguish important from unimportant features but should be interpreted cautiously for precise quantitative claims.

This interpretation acknowledges both strengths (correlation above threshold) and limitations (53\% explained variance leaves 47\% unexplained). Honest assessment builds trust with legal professionals who will rely on these reports.

\subsubsection{Field 4: Confidence Interval Calibration}

\textbf{Example (abbreviated):}

\texttt{CI CALIBRATION}

\textit{Coverage Rate:} 91.3\% (913 of 1,000 within predicted 90\% CI)

\textit{Calibration Test:} Binomial p~=~0.42 (fail to reject $H_0: p_{\text{coverage}} = 0.90$) $\Rightarrow$ \textbf{well-calibrated}

\textit{Interpretation:} Confidence intervals are reliable. Observed coverage (91.3\%) closely matches nominal 90\%. Practitioners can trust that reported CIs will contain true values $\sim$90\% of the time. Slight over-coverage (91.3\% vs.\ 90\%) suggests conservative (wider) intervals—acceptable in forensic contexts.

\subsubsection{Field 5: Known Error Rates and Failure Modes}

This is often the most important field for legal professionals. It directly addresses Daubert's error rate requirement.

\textbf{Example:}

\texttt{KNOWN ERROR RATES}

\textit{Overall Falsification Rate:} 38\% (380 of 1,000 test cases FALSIFIED), 95\% CI: [35.1\%, 40.9\%]

\textit{Failure Modes:}
\begin{itemize}
\item Non-Triviality: 2.1\% (21 cases)
\item Insufficient Statistical Evidence: 35.9\% (359 cases)
\item Separation Margin: 0\% (by design)
\end{itemize}

\textit{Demographic Stratification:}

\begin{table}[H]
\centering
\small
\begin{tabular}{lcc}
\toprule
\textbf{Group} & \textbf{N} & \textbf{Falsif. Rate} \\ \midrule
\multicolumn{3}{l}{\textit{Age}} \\
~~Young ($<$30y) & 287 & 34\% \\
~~Middle (30--50y) & 485 & 37\% \\
~~Older ($>$50y) & 228 & 45\% $\dagger$ \\ \midrule
\multicolumn{3}{l}{\textit{Gender}} \\
~~Male & 768 & 36\% \\
~~Female & 232 & 42\% \\ \midrule
\multicolumn{3}{l}{\textit{Skin Tone}} \\
~~Light & 831 & 35\% \\
~~Dark & 169 & 43\% \\ \bottomrule
\multicolumn{3}{l}{\footnotesize $\dagger$ HIGH DISPARITY: 11pp gap (older vs.\ young)}
\end{tabular}
\end{table}

\textit{Known Failure Scenarios:}
\begin{enumerate}
\item Extreme poses ($>$45$^\circ$ rotation): 52\% falsification rate
\item Heavy occlusion (surgical masks, hands covering face): 61\%
\item Low resolution ($<$80$\times$80 pixels): 48\%
\item Older individuals ($>$50 years): 45\% (age bias)
\end{enumerate}

\textit{Interpretation:} Method achieves NOT FALSIFIED status for 62\% of cases but exhibits systematic biases. Higher failure rates for older individuals, females, and darker skin tones indicate demographic disparities. Use with caution in forensically diverse contexts; restrict to high-quality frontal images; require mandatory demographic audit.

This honest reporting of failures builds credibility. Legal professionals can assess whether the method is appropriate for their specific case demographics.

\subsubsection{Field 6: Limitations and Scope}

\textbf{Example (abbreviated):}

\texttt{LIMITATIONS}

\textit{Dataset:} Validated on LFW (celebrity images, frontal poses, high resolution). May NOT generalize to surveillance footage, infrared imagery, or non-Western demographics.

\textit{Model:} ArcFace ResNet-100 specific. Results may differ for CosFace, transformer models, or different embedding dimensions.

\textit{Out-of-Scope:} Video, 3D faces, face identification (1:N search), adversarial robustness, real-time deployment ($\sim$4--9 seconds per image prohibits real-time use).

These limitations aren't weaknesses to hide—they define the scope within which claims hold. Transparent acknowledgment prevents misuse.

\subsubsection{Field 7: Recommendation and Confidence Assessment}

This field translates technical findings into actionable guidance.

\textbf{Example (Moderate Performance with Restrictions):}

\texttt{RECOMMENDATION}

\textit{Verdict:} \textbf{NOT FALSIFIED}

\textit{Confidence Level:} \textbf{MODERATE} (correlation $\rho=0.73$ above threshold, but 38\% falsification rate and demographic disparities)

\textit{Deployment Recommendation:} \textbf{APPROVED for forensic use with RESTRICTIONS}

\textit{Mandatory Restrictions:}
\begin{enumerate}
\item Image quality: Minimum 100$\times$100 pixels, pose $<$30$^\circ$ rotation, no heavy occlusion
\item Demographic audit: Report stratified performance for each case's demographic category
\item Human expert review: Required when attributions highlight unusual regions (e.g., $>$30\% importance on background)
\item Uncertainty disclosure: Always report 90\% confidence intervals
\item Evidentiary limitation: Use as investigative aid, NOT sole evidence; require corroboration
\end{enumerate}

\textit{Contraindications (DO NOT USE):}
\begin{itemize}
\item Surveillance footage $<$80$\times$80 pixels
\item Profile views ($>$30$^\circ$ rotation)
\item Video-based verification
\item Real-time deployment
\end{itemize}

\textit{Justification:} Moderate predictive accuracy ($\rho=0.73$, R$^2=0.53$) and well-calibrated uncertainty (91.3\% coverage) indicate attributions provide useful forensic insights. However, 38\% falsification rate and demographic disparities (11pp gap for age) necessitate restrictions. These balance utility (enabling use where validation is strongest) with safety (preventing misuse in scenarios where validation fails).

\subsection{Example Completed Report}
\label{sec:example_report}

Due to space constraints, we present an abbreviated complete report demonstrating the template in practice. Full examples with all fields are available in supplementary materials.

\begin{table}[H]
\centering
\scriptsize
\caption{Hypothetical Forensic Attribution Validation Report (Abbreviated)}
\begin{tabular}{@{}p{0.95\linewidth}@{}}
\toprule
\textbf{FORENSIC ATTRIBUTION VALIDATION REPORT} \\
\textit{Case ID:} [Redacted] \quad \textit{Date:} 2024-10-20 \quad \textit{Analyst:} [Name, Credentials] \\ \midrule
\textbf{Field 1: Method ID} \\
Grad-CAM (Captum v0.6.0) | ArcFace ResNet-100 (VGGFace2-HQ, official release) \\ \midrule
\textbf{Field 2: Parameters} \\
$\theta_{\text{high}}=0.7$, $\theta_{\text{low}}=0.4$ | $\delta_{\text{target}}=0.8$~rad, K=200 | Pre-reg: OSF \texttt{[ID]}, 2024-10-15 | LFW, N=1,000 \\ \midrule
\textbf{Field 3: $\Delta$-Accuracy} \\
$\rho=0.73$ [0.68, 0.78], p=0.012 (\textbf{MET}) | R$^2$=0.53 | MAE=0.11~rad (6.3$^\circ$) \\ \midrule
\textbf{Field 4: CI Calibration} \\
Coverage: 91.3\%, binomial p=0.42 (\textbf{well-calibrated}) \\ \midrule
\textbf{Field 5: Error Rates} \\
Falsif.: 38\% [35.1\%, 40.9\%] | Age: 34\%/37\%/45\% (young/mid/older, \textbf{11pp disparity}) | Failure: pose$>$30$^\circ$ (52\%), occlusion (61\%) \\ \midrule
\textbf{Field 6: Limitations} \\
LFW (celebrity, frontal, high-res) | ArcFace-specific | Out-of-scope: video, 3D, real-time \\ \midrule
\textbf{Field 7: Recommendation} \\
Verdict: \textbf{NOT FALSIFIED} | Confidence: \textbf{MODERATE} | Deployment: \textbf{APPROVED with RESTRICTIONS} \\
Restrictions: (1) Image quality $\geq$100px, pose$<$30$^\circ$; (2) Demographic audit; (3) Expert review for unusual attributions; (4) Report 90\% CIs; (5) Investigative aid only, require corroboration \\
Contraindications: Surveillance$<$80px, profile views, video, real-time \\ \midrule
\textit{Analyst Signature:} [Signature] \quad \textit{Supervisor:} [Signature] \quad \textit{Date:} 2024-10-20 \\ \bottomrule
\end{tabular}
\end{table}

This condensed format is suitable for case files. The full report (with visualizations, detailed tables, statistical test outputs) would run 10--15 pages.

\subsection{Usage Guidance for Practitioners}

\textbf{When to complete this template:}
\begin{enumerate}
\item Before deploying a new attribution method in forensic investigations
\item After model updates (retraining, fine-tuning)
\item When dataset shifts significantly (e.g., surveillance footage after validating on LFW)
\item Annual review (periodic revalidation)
\end{enumerate}

\textbf{Legal and ethical considerations:}
\begin{itemize}
\item \textit{Daubert compliance:} Field~1 (peer review), Field~3 (testability), Field~5 (error rates)
\item \textit{GDPR/AI Act:} Field~2 (logic), Field~3 (accuracy), Field~6 (transparency)
\item \textit{Transparency:} Always disclose template to defendants, legal counsel, oversight bodies
\item \textit{Disclosure:} Make available upon FOIA/public records requests; include as exhibit in court
\end{itemize}

\textbf{Template versioning:} This is version 1.0 (2024). As regulatory frameworks evolve or new scientific evidence emerges, we will update field requirements. Check \texttt{[repository URL]} for the latest version.
